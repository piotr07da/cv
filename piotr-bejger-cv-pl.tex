\documentclass[a4paper,11pt]{article}

\usepackage{xcolor}
\usepackage{multicol}
\usepackage{fontspec}
\usepackage[a4paper,margin=0cm]{geometry}
\usepackage[polish]{babel} %polish lang
\usepackage{polski} %lamanie wyrazow jezyka polskiego
%\usepackage{showframe}

\setmainfont{Segoe UI}
\newfontfamily{\segoesym}{Segoe UI Symbol}

\definecolor{fg0}{RGB}{12, 20, 18}
\definecolor{bg0}{RGB}{255, 255, 255}
\definecolor{fg1}{RGB}{248, 250, 249}
\definecolor{bg1}{RGB}{82, 128, 120}
% w celu wyłączenia kolorów trzeba odkomentować poniższe linie
%\colorlet{fg1}{fg0}
%\colorlet{bg1}{bg0}

\renewcommand{\labelitemi}{$►$}

\newcommand{\switchColors}[0]
{
	\colorlet{tmpfg}{fg0}
	\colorlet{tmpbg}{bg0}
	\colorlet{fg0}{fg1}
	\colorlet{bg0}{bg1}
	\colorlet{fg1}{tmpfg}
	\colorlet{bg1}{tmpbg}
	\colorlet{fg}{fg0}
	\colorlet{bg}{bg0}
}

\newcommand{\cvEnuItemBox}[1]
{%
	\hspace{26pt}\parbox{.93\textwidth}{#1}
}

\newcommand{\cvSectionHeader}[1]
{%
	\noindent\setlength{\fboxsep}{0pt}\setlength{\fboxrule}{0pt}%
	\colorbox{bg}
	{%
		\color{fg}%
		\begin{minipage}[t]{\textwidth}
			\begin{flushright}\vspace{12pt}
			\vspace{6pt}%
			{\fontsize{19pt}{1em}\addfontfeature{LetterSpace=7.0}\selectfont #1\hspace*{16pt}}\vspace{2pt}
			\vspace{4pt}%
			\end{flushright}%
		\end{minipage}%
	}%
}

\newcommand{\cvSectionEntry}[1]
{%
	\vspace{-1pt}\noindent\setlength{\fboxsep}{0pt}\setlength{\fboxrule}{0pt}%
	\colorbox{bg}
	{%
		\color{fg}%
		\begin{minipage}[t]{\textwidth}
			#1 %
		\end{minipage}%
	}%
}

\newcommand{\cvSectionEntryHeader}[4]
{%
	\vspace{8pt}\hspace{26pt}\parbox{.93\textwidth}
	{%
		\fontsize{13pt}{1em}\selectfont #1 \hspace{2pt} #2 \hfill #3 {\segoesym ➔} #4 %
	}\\
}

\newcommand{\cvSectionEntryBody}[2]
{%
	\cvEnuItemBox{\hspace{27pt}#1#2}
}



\begin{document}

\switchColors

\vspace*{-22pt}\noindent\colorbox{bg}
{
	\color{fg}
	\begin{minipage}[t]{\textwidth}
		\vspace{20pt}
			\begin{minipage}[c]{0.37\textwidth}
				\hfill\vfill
			\end{minipage} %
			\begin{minipage}[c]{0.27\textwidth}
				{\fontsize{50pt}{1em}\addfontfeature{LetterSpace=9.0}\selectfont PIOTR} \vspace{11pt}\\
				{\fontsize{16pt}{1em}\addfontfeature{LetterSpace=42.5}\selectfont 731-444-160}\\
				\hfill\vfill
			\end{minipage} %
			\begin{minipage}[c]{0.01\textwidth}
				\hfill\vfill
			\end{minipage} %
			\begin{minipage}[c]{0.34\textwidth}
				{\fontsize{16pt}{1em}\addfontfeature{LetterSpace=5.2}\selectfont piotr.bejger@gmail.com} \vspace{11pt}\\
				{\fontsize{50pt}{1em}\addfontfeature{LetterSpace=9.0}\selectfont BEJGER}\\
				\hfill\vfill
			\end{minipage} %
			\begin{minipage}[c]{0.01\textwidth}
				\hfill\vfill
			\end{minipage} %
		\vspace{8pt}
	\end{minipage}
}

% PODSUMOWANIE KWALIFIKACJI
\switchColors
\cvSectionHeader{Podsumowanie kwalifikacji}

\cvSectionEntry
{	
	\cvEnuItemBox
	{
		\begin{itemize}
			\setlength\itemsep{-1pt}
			\item Programowanie z dużym naciskiem na dobrą jakość kodu, zasady \textbf{SOLID} oraz inne dobre praktyki.
			\item Znajomość języków: \textbf{C\#} oraz Sql, VB, Dart, C/C++, Js, Html, Css.
			\item Doświadczenie w \textbf{OOP}.
			\item Doświadczenie w \textbf{TDD}.
			\item Znajomość: \textbf{DDD, CQRS, Event Sourcing}.
			\item Znajomość: \textbf{.NET} (framework/core), WPF (MVVM), WCF, WF, ASP.NET.
			\item Znajomość: \textbf{Azure DevOps, Git}.
			\item Doświadczenie w pracy w oparciu o \textbf{Scrum/Agile}.
		\end{itemize}
	}%
}

% DOSWIADCZENIE ZAWODOWE
\switchColors
\cvSectionHeader{Doświadczenie zawodowe}

% ProService Finteco
\cvSectionEntry
{
	\cvSectionEntryHeader{❹}{ProService Finteco Sp. z o.o. (senior .NET developer / tech lead)}{2017-03}{\cdot\cdot\cdot}

	\cvSectionEntryBody{Wybrane osiągnięcia:}
	{
		\begin{itemize}
			\setlength\itemsep{-1pt}
			\item Objęcie roli tech leadera w zespole.
			\item Istotne zwiększenie jakości kodu poprzez wdrażanie zasad \textbf{SOLID} i innych dobrych praktyk.
			\item Usunięcie około 70\% legacy codu przy zachowaniu pełnej funkcjonalności systemu, dzięki czemu znacząco wzrosła utrzymywalność kodu i zmalał koszt developmentu kolejnych funkcjonalności.
			\item Stworzenie mechanizmów i narzędzi do ułatwienia zarządzania konfiguracją systemu zarówno na środowiskach developerskich, jak i produkcyjnych. Dzięki temu uzyskano sprawniejszy i mniej ryzykowny deployment.
		\end{itemize}
	}
}

% Bonair
\cvSectionEntry
{
	\cvSectionEntryHeader{❸}{BONAIR S.A. (senior .NET developer)}{2013-07}{2017-02}

	\cvSectionEntryBody{Wybrane osiągnięcia:}
	{
		\begin{itemize}
			\setlength\itemsep{-1pt}
			\item Rozwój i modernizacja systemu CasePro (system do modelowania procesów biznesowych).
			\item Istotne zwiększenie jakości kodu poprzez wdrażanie zasad \textbf{SOLID} i innych dobrych praktyk.
			\item Wdrożenie \textbf{testów jednostkowych}. Wdrożenie \textbf{TDD} w pewnych obszarach systemu CasePro.
			\item Projekt i implementacja dedykowanego mechanizmu logowania przebiegu procesów biznesowych, stosowanego do analizy wydajności i prawidłowości ich działania. [\textbf{C\#, .NET, MsSql}]
			\item Integracja systemów zewnętrznych KRD, BIK, Asseco DEF z CasePro. [\textbf{WF, WCF}]
			\item Stworzenie aplikacji do wysyłania JPK (jednolitych plików kontrolnych). [\textbf{WPF}]
			\item Stworzenie aplikacji pozwalającej rejestrować zdarzenia komunikacyjne w Krakowie dla MPK Kraków. [\textbf{WPF, WCF}]
		\end{itemize}
	}
	\vspace{53mm}
}

\pagebreak\vspace*{-11pt}

% Oponeo.pl
\cvSectionEntry
{
	\cvSectionEntryHeader{❷}{OPONEO.PL S.A. (.NET developer)}{2010-01}{2013-06}

	\cvSectionEntryBody{Wybrane osiągnięcia:}
	{
		\begin{itemize}
			\setlength\itemsep{-1pt}
			\item Stworzenie od podstaw sklepu Felgi.pl. Implementacja w języku \textbf{C\#} w technologii \textbf{.NET} 4.0, w oparciu o autorski framework aplikacji web. Projekt i implementacja wszystkich algorytmów warstwy logiki biznesowej. [\textbf{C\#} i \textbf{Sql}]
			\item Optymalizacja sklepów Oponeo.pl, Opony.com i innych -- 10-krotne zwiększenie liczby obsługiwanych użytkowników. Narzędzia wykorzystane w celach diagnostycznych to \textbf{ANTS Performance} i \textbf{Memory Profiler}.
			\item Wdrożenie \textbf{testów jednostkowych}. Zastosowany framework to \textbf{NUnit}.
			\item Implementacja narzędzia analitycznego, rejestrującego akcje użytkowników na stronach (ruchy myszką, kliknięcia, nawigacja pomiędzy stronami). [\textbf{WCF, Ajax, jQuery}]
			\item Zbudowanie modułu komunikacji pomiędzy instancjami aplikacji działającymi w środowisku opartym o load balancing. W tym celu wykorzystanie \textbf{WCF} oraz \textbf{iControl API} (infrastruktura oparta o BIG-IP LTM firmy \textbf{F5}).
			\item Zaprojektowanie i implementacja narzędzia [\textbf{C\#, WCF, Silverlight}] do monitorowania stanu maszyn wirtualnych przy pomocy \textbf{WMI} oraz zarządzania serwerem \textbf{IIS} przy pomocy biblioteki Microsoft.Web.Administration.
			\item Wprowadzenie elementów programowania aspektowego (\textbf{AOP}) w kontekście walidacji argumentów metod. Rozwiązanie oparte o \textbf{Mono.Cecil}, pozwala na ``wstrzykiwanie`` kodu walidującego w języku \textbf{MSIL}.
		\end{itemize}
	}
}

% T Komp
\cvSectionEntry
{
	\cvSectionEntryHeader{❶}{T Komp (junior developer)}{2009-03}{2009-12}

	\cvSectionEntryBody{Wybrane osiągnięcia:}
	{
		\begin{itemize}
			\setlength\itemsep{-1pt}
			\item Stworzenie (analiza, projekt i implementacja) systemu wspomagającego proces wytwarzania opakowań w firmie Reckitt Benckiser. [ASP, VB, JavaScript, \textbf{Sql}]
			\item Praca nad rozwojem NND Dynamic (jako główny developer) jednego z modułów systemu NND (flagowy produkt firmy).
			\item Realizacja projektów: ``Suppliers Assortment``, ``Trainings`` oraz ``Holiday Request`` dla Reckitt Benckiser; ``Szkolenia BHP`` dla Mondi Świecie; ``Formularz Zgłoszeniowy`` dla Smurfit Kappa.
			\item Główny udział w tworzeniu systemu EOD dla firmy Nutricia. Projekt obejmował realizację procesu rejestracji kontraktów, zamówień oraz faktur.
		\end{itemize}
	}
}

% Centrum Onkologii
\cvSectionEntry
{
	\cvSectionEntryHeader{⓿}{Centrum Onkologii w Bydgoszczy (sekretarz medyczny)}{2005-04}{2009-02}

	\cvSectionEntryBody{Zakres obowiązków:}
	{
		\begin{itemize}
			\setlength\itemsep{0em}
			\item Nadzór dokumentacji medycznej.
		\end{itemize}
	}
}

% WYKSZTALCENIE
\switchColors
\cvSectionHeader{Wykształcenie}

% Politechnika Poznanska
\cvSectionEntry
{
	\cvSectionEntryHeader{❶}{Politechnika Poznańska}{2008}{2010}

	\cvSectionEntryBody{}
	{
		\begin{itemize}
			\setlength\itemsep{-1pt}
			\item Wydział Informatyki i Zarządzania • Kierunek: Informatyka
			\item Uzyskany tytuł: magister (inżynieria komputerowa)
			\item Praca magisterska: \textit{Implementacja teorii zbiorów przybliżonych w technologii Silverlight}
		\end{itemize}
	}
}

% Uniwersytet Kazimierza Wielkiego
\cvSectionEntry
{
	\cvSectionEntryHeader{⓿}{Uniwersytet Kazimierza Wielkiego w Bydgoszczy}{2004}{2008}

	\cvSectionEntryBody{}
	{
		\begin{itemize}
			\setlength\itemsep{-1pt}
			\item Wydział Matematyki, Fizyki i Techniki • Kierunek: Informatyka
			\item Uzyskany tytuł: inżynier
			\item Praca inżynierska: \textit{Wykorzystanie sztucznej sieci neuronowej do prognozowania kursów walut}
		\end{itemize}
	}
}

\pagebreak\vspace*{-12pt}

% DODATKOWE INFORMACJE
\switchColors
\cvSectionHeader{Dodatkowe informacje}

\cvSectionEntry
{
	\cvEnuItemBox
	{
		\begin{itemize}
			\setlength\itemsep{-1pt}
			\item Znajomość języka angielskiego w stopniu komunikatywnym.
			\item Amatorskie programowanie mikrokontrolerów AVR.
			\item Tworzenie gier mobilnych opartych o silnik Unity.
			\item Certyfikaty Microsoft: MCSD, MCP, MS (2014).
			\item Ukończony kurs Cisco Certified Network Associate (2008 {\segoesym ➔} 2009).
			\item Prawo jazdy kategorii B.
			\item Zainteresowania: strzelectwo (International Defensive Pistol Association), trójbój siłowy.
		\end{itemize}
	}
}

\center
{
	\parbox{.9\textwidth}
	{
		\center
		{
			\fontsize{9pt}{1em}\selectfont
			Wyrażam zgodę na przetwarzanie moich danych osobowych dla potrzeb niezbędnych do realizacji procesu rekrutacji.\\
			Wyrażam zgodę na przetwarzanie moich danych osobowych w zakresie przyszłych procesów rekrutacyjnych.
		}
	}
}

\vfill

\end{document}