\documentclass[a4paper,11pt]{report}

\usepackage{xcolor}
\usepackage{multicol}
\usepackage{fontspec}
\usepackage[a4paper,margin=0cm]{geometry}
%\usepackage{showframe}

\setmainfont{Segoe UI}
\newfontfamily{\segoesym}{Segoe UI Symbol}

\definecolor{titlefg}{RGB}{250, 247, 251}
\definecolor{titlebg}{RGB}{104, 42, 123}
\definecolor{sectionfg0}{RGB}{0, 0, 0}
\definecolor{sectionbg0}{RGB}{255, 255, 255}
\definecolor{sectionfg1}{RGB}{248, 250, 249}
\definecolor{sectionbg1}{RGB}{82, 128, 120}

\addtolength{\voffset}{-8pt}
\addtolength{\hoffset}{-17pt}

\renewcommand{\labelitemi}{$►$}

\newcommand{\cvEnuItemBox}[1]
{%
	\hspace{26pt}\parbox{.93\textwidth}{#1}
}

\newcommand{\cvSectionHeader}[1]
{%
	\setlength{\fboxsep}{0pt}%
	\setlength{\fboxrule}{0pt}%
	\colorbox{sectionbg}
	{%
		\color{sectionfg}%
		\begin{minipage}[t]{\textwidth}
			\begin{flushright}\vspace{12pt}
			{\fontsize{19pt}{1em}\addfontfeature{LetterSpace=7.0}\selectfont #1\hspace*{16pt}}\vspace{2pt}
			\end{flushright}%
		\end{minipage}%
	}%
}

\newcommand{\cvSectionEntry}[1]
{%
	\setlength{\fboxsep}{0pt}%
	\setlength{\fboxrule}{0pt}%
	\colorbox{sectionbg}
	{%
		\color{sectionfg}%
		\begin{minipage}[t]{\textwidth}
			#1 %
		\end{minipage}%
	}%
}

\newcommand{\cvSectionEntryHeader}[4]
{%
	\hspace{26pt}\parbox{.93\textwidth}
	{%
		\fontsize{13pt}{1em}\selectfont #1 \hspace{2pt} #2 \hfill #3 {\segoesym ➔} #4 %
	}\\
}

\newcommand{\cvSectionEntryBody}[2]
{%
	\cvEnuItemBox{\hspace{27pt}#1#2}
}



\begin{document}

\colorbox{titlebg}
{
	\color{titlefg}
	\begin{minipage}[t]{\textwidth}
		\vspace{20pt}
			\begin{minipage}[c]{0.37\textwidth}
				\hfill\vfill
			\end{minipage} %
			\begin{minipage}[c]{0.27\textwidth}
				{\fontsize{50pt}{1em}\addfontfeature{LetterSpace=9.0}\selectfont PIOTR} \vspace{11pt}\\
				{\fontsize{16pt}{1em}\addfontfeature{LetterSpace=42.5}\selectfont 731-444-160}\\
				\hfill\vfill
			\end{minipage} %
			\begin{minipage}[c]{0.01\textwidth}
				\hfill\vfill
			\end{minipage} %
			\begin{minipage}[c]{0.34\textwidth}
				{\fontsize{16pt}{1em}\addfontfeature{LetterSpace=5.2}\selectfont piotr.bejger@gmail.com} \vspace{11pt}\\
				{\fontsize{50pt}{1em}\addfontfeature{LetterSpace=9.0}\selectfont BEJGER}\\
				\hfill\vfill
			\end{minipage} %
			\begin{minipage}[c]{0.01\textwidth}
				\hfill\vfill
			\end{minipage} %
		\vspace{8pt}
	\end{minipage}
}

\colorlet{sectionfg}{sectionfg0}
\colorlet{sectionbg}{sectionbg0}

% PODSUMOWANIE KWALIFIKACJI

\cvSectionHeader{Podsumowanie kwalifikacji}

\cvSectionEntry
{	
	\cvEnuItemBox
	{
		\begin{itemize}
			\setlength\itemsep{0em}
			\item Znajomość języków: \textbf{C\#} (oraz Sql, Dart, C/C++, Js, Html, Css)
			\item Doświadczenie w \textbf{OOP}
			\item Doświadczenie w \textbf{TDD}
			\item Znajomość: \textbf{DDD}, \textbf{Event Sourcing}
			\item Znajomość: \textbf{.NET} (framework/core), WPF (MVVM), WCF, WF, ASP.NET
			\item Znajomość: Azure DevOps, Git
		\end{itemize}
	}%
}

\colorlet{sectionfg}{sectionfg1}
\colorlet{sectionbg}{sectionbg1}

% DOSWIADCZENIE ZAWODOWE

\cvSectionHeader{Doświadczenie zawodowe}

% ProService Finteco

\cvSectionEntry
{
	\cvSectionEntryHeader{❹}{ProService Finteco Sp. z o.o. (senior .NET developer / tech lead)}{2017-03}{\cdot\cdot\cdot}

	\cvSectionEntryBody{Wybrane osiągnięcia:}
	{
		\begin{itemize}
			\setlength\itemsep{0em}
			\item one
			\item two
		\end{itemize}
	}
}

% Bonair

\cvSectionEntry
{
	\cvSectionEntryHeader{❸}{BONAIR S.A. (senior .NET developer)}{2013-07}{2017-02}

	\cvSectionEntryBody{Wybrane osiągnięcia:}
	{
		\begin{itemize}
			\setlength\itemsep{0em}
			\item Projekt i implementacja dedykowanego mechanizmu logowania przebiegu procesów biznesowych, stosowanego do analizy wydajności i prawidłowości ich działania. (C\#, .NET, MSSQL)
		\end{itemize}
	}
}

% Oponeo.pl

\cvSectionEntry
{
	\cvSectionEntryHeader{❷}{OPONEO.PL S.A. (.NET developer)}{2010-01}{2013-06}

	\cvSectionEntryBody{Wybrane osiągnięcia:}
	{
		\begin{itemize}
			\setlength\itemsep{0em}
			\item Stworzenie od podstaw sklepu Felgi.pl. Implementacja w języku \textbf{C\#} w technologii \textbf{.NET 4.0}, w oparciu o autorski framework aplikacji web. Projekt i implementacja wszystkich algorytmów warstwy logiki biznesowej (\textbf{C\#} i \textbf{Sql}).
			\item Optymalizacja aplikacji Oponeo.pl, Opony.com i innych. Całość wprowadzonych zabiegów optymalizacyjnych, pozwoliła na 10-krotne zwiększenie liczby obsługiwanych użytkowników. Narzędzia wykorzystane w celach diagnostycznych to \textbf{ANTS Performance} i \textbf{Memory Profiler}.
			\item Wdrożenie testów jednostkowych. Zastosowany framework to \textbf{NUnit}.
			\item Implementacja narzędzia analitycznego, rejestrującego akcje użytkowników na stronach (ruchy myszką, kliknięcia, przejścia pomiędzy stronami). Implementacja oparta o \textbf{WCF}, \textbf{Ajax}, \textbf{jQuery}.
			\item Zbudowanie modułu komunikacji pomiędzy instancjami aplikacji w środowisku opartym o load balancing. W tym celu wykorzystanie technologii \textbf{WCF} oraz \textbf{iControl API} (infrastruktura oparta o BIG-IP LTM firmy \textbf{F5}).
			\item Zaprojektowanie i implementacja narzędzia do monitoringu infrastruktury (\textbf{C\#}, \textbf{WCF}, \textbf{Silverlight}). Monitorowanie stanu maszyn wirtualnych przy pomocy \textbf{WMI}. Monitorowanie i zarządzanie serwerem IIS przy pomocy biblioteki Microsoft.Web.Administration.
			\item Wprowadzenie elementów programowania aspektowego (\textbf{AOP}) w kontekście walidacji argumentów metod. Rozwiązanie oparte o \textbf{Mono.Cecil}, pozwala na "wstrzykiwanie" odpowiednich fragmentów kodu walidującego w języku \textbf{MSIL}.
		\end{itemize}
	}
}

% T Komp

\cvSectionEntry
{
	\cvSectionEntryHeader{❶}{T Komp (junior developer)}{2009-03}{2009-12}

	\cvSectionEntryBody{Wybrane osiągnięcia:}
	{
		\begin{itemize}
			\setlength\itemsep{0em}
			\item Stworzenie (analiza, projekt, implementacja) systemu wspomagającego proces wytwarzania opakowań w firmie Reckitt Benckiser (ASP, VB, JavaScript, Sql).
			\item Praca nad rozwojem NND Dynamic (jako główny programista), jednego z modułów systemu NND (flagowy produkt firmy).
			\item Realizacja projektów: “Suppliers Assortment”, “Trainings” oraz “Holiday Request” dla Reckitt Benckiser; “Szkolenia BHP” dla Mondi Świecie; “Formularz Zgłoszeniowy” dla Smurfit Kappa.
			\item Główny udział w tworzeniu systemu EOD dla firmy Nutricia. Projekt obejmował realizację procesu rejestracji kontraktów, zamówień oraz faktur.
		\end{itemize}
	}
}

% Centrum Onkologii

\cvSectionEntry
{
	\cvSectionEntryHeader{⓿}{Centrum Onkologii w Bydgoszczy (sekretarz medyczny)}{2005-04}{2009-02}

	\cvSectionEntryBody{Zakres obowiązków:}
	{
		\begin{itemize}
			\setlength\itemsep{0em}
			\item Nadzór dokumentacji medycznej.
		\end{itemize}
	}
}



\cvSectionHeader{Wykształcenie}

\cvSectionEntryHeader{❶}{Politechnika Poznańska}{2008}{2010}

\cvSectionEntryBody{}
{
	\begin{itemize}
		\setlength\itemsep{0em}
		\item Wydział Informatyki i Zarządzania • Kierunek: Informatyka
		\item Uzyskany tytuł: magister (inżynieria komputerowa)
		\item Praca magisterska: \textit{Implementacja teorii zbiorów przybliżonych w technologii Silverlight}
	\end{itemize}
}

\cvSectionEntryHeader{⓿}{Uniwersytet Kazimierza Wielkiego w Bydgoszczy}{2004}{2008}

\cvSectionEntryBody{}
{
	\begin{itemize}
		\setlength\itemsep{0em}
		\item Wydział Matematyki, Fizyki i Techniki • Kierunek: Informatyka
		\item Uzyskany tytuł: inżynier
		\item Praca inżynierska: \textit{Wykorzystanie sztucznej sieci neuronowej do prognozowania kursów walut}
	\end{itemize}
}

\cvSectionHeader{Dodatkowe informacje}

\cvEnuItemBox
{
	\begin{itemize}
		\setlength\itemsep{0em}
		\item Znajomość języka angielskiego w stopniu komunikatywnym
		\item Tworzenie gier mobilnych opartych o silnik Unity
		\item Amatorskie programowanie mikrokontrolerów AVR
		\item Certyfikaty Microsoft: MCSD, MCP, MS (2014)
		\item Ukończony kurs Cisco Certified Network Associate (2008 {\segoesym ➔} 2009)
		\item Prawo jazdy kategorii B
	\end{itemize}
}

\center
{
	\parbox{.9\textwidth}
	{
		\center
		{
			\fontsize{9pt}{1em}\selectfont
			Wyrażam zgodę na przetwarzanie moich danych osobowych zawartych w ofercie pracy dla potrzeb procesu rekrutacji zgodnie z ustawą z dnia 27.08.1997r. Dz. U. z 2002 r., Nr 101, poz. 923 ze zm.
		}
	}
}

\vfill

\end{document}